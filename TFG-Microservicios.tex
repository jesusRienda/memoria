%---- IMPORTANTE ----
% Esta plantilla está basada en las recomendaciones de la guía "Trabajo fin de Grado: Escribir el TFG", que encontrarás en http://uc3m.libguides.com/TFG/escribir
% contiene recomendaciones de la Biblioteca basadas principalmente en estilos APA e IEEE, pero debes seguir siempre las orientaciones de tu Tutor de TFG y la normativa de TFG para tu titulación.
% Encontrarás un ejemplo de TFG realizado con esta misma plantilla en el archivo "ejemplo_TFG_2017.zip", incluido en la misma carpeta. Consúltalo porque contiene ejemplos útiles para incorporar tablas, figuras, listados de código, bibliografía, etc.


%----------
%	CONFIGURACIÓN DEL DOCUMENTO
%----------

\documentclass[12pt]{report} %fuente a 12pt

% MÁRGENES: 2,5 cm sup. e inf.; 3 cm izdo. y dcho.
\usepackage[
a4paper,
vmargin=2.5cm,
hmargin=3cm
]{geometry}

% INTERLINEADO: Estrecho (6 ptos./interlineado 1,15) o Moderado (6 ptos./interlineado 1,5)
\renewcommand{\baselinestretch}{1.15}
\parskip=6pt

% soporte para generar PDF/A --es importante de cara a su inclusión en e-Archivo porque es el formato óptimo de preservación y a la generación de metadatos, tal y como se describe en http://uc3m.libguides.com/ld.php?content_id=31389625. En la carpeta incluímos el archivo plantilla_tfg_2017.xmpdata en el que puedes incluir los metadatos que se incorporarán al archivo PDF cuando lo compiles. Ese archivo debe llamarse igual que tu archivo .tex
\usepackage[a-1b]{pdfx}

\usepackage{hyperref}
\hypersetup{linktoc=all}

% Expresiones matemáticas
\usepackage{amsmath,amssymb,amsfonts,amsthm}

\usepackage{txfonts} 
\usepackage[T1]{fontenc}
\usepackage[utf8]{inputenc}

\usepackage[spanish, es-tabla]{babel} % información sobre el paquete babel para español http://osl.ugr.es/CTAN/language/spanish/babel/base/spanish.pdf
\usepackage[babel, spanish=spanish]{csquotes}
\AtBeginEnvironment{quote}{\small}

% DEFINICIÓN DE COLORES para portada y listados de código
\usepackage{color}
\definecolor{azulUC3M}{RGB}{0,0,102}
\definecolor{gray97}{gray}{.97}
\definecolor{gray75}{gray}{.75}
\definecolor{gray45}{gray}{.45}

% diseño de PIE DE PÁGINA
\usepackage{fancyhdr}
\pagestyle{fancy}
\fancyhf{}
\renewcommand{\headrulewidth}{0pt}
\rfoot{\thepage}
\fancypagestyle{plain}{\pagestyle{fancy}}

% DISEÑO DE LOS TÍTULOS de las partes del trabajo (capítulos y epígrafes o subcapítulos)
\usepackage{titlesec}
\usepackage{titletoc}
\titleformat{\chapter}[block]
	{\large\bfseries\filcenter}
	{\thechapter.}
	{5pt}
	{\MakeUppercase}
	{}
\titlespacing{\chapter}{0pt}{0pt}{*3}
\titlecontents{chapter}
	[0pt]                                               
	{}
	{\contentsmargin{0pt}\thecontentslabel.\enspace\uppercase}
	{\contentsmargin{0pt}\uppercase}                        
	{\titlerule*[.7pc]{.}\contentspage}                 
  
\titleformat{\section}
	{\bfseries}
	{\thesection.}
	{5pt}
	{}
\titlecontents{section}
	[5pt]                                               
	{}
	{\contentsmargin{0pt}\thecontentslabel.\enspace}
	{\contentsmargin{0pt}}
	{\titlerule*[.7pc]{.}\contentspage}

\titleformat{\subsection}
	{\normalsize\bfseries}
	{\thesubsection.}
	{5pt}
	{}
\titlecontents{subsection}
	[10pt]                                               
	{}
	{\contentsmargin{0pt}                          
		\thecontentslabel.\enspace}
	{\contentsmargin{0pt}}                        
	{\titlerule*[.7pc]{.}\contentspage}  


% DISEÑO DE TABLAS. Puedes elegir entre el estilo para ingeniería o para ciencias sociales y humanidades. Por defecto, está activado el estilo de ingeniería. Si deseas utilizar el otro, comenta las líneas del diseño de ingeniería y descomenta las del diseño de ciencias sociales y humanidades
\usepackage{multirow} % permite combinar celdas 
\usepackage{caption} % para personalizar el título de tablas y figuras
\usepackage{floatrow} % utilizamos este paquete y sus macros \ttabbox y \ffigbox para alinear los nombres de tablas y figuras de acuerdo con el estilo definido. Para su uso ver archivo de ejemplo 
\usepackage{array} % con este paquete podemos definir en la siguiente línea un nuevo tipo de columna para tablas: ancho personalizado y contenido centrado
\newcolumntype{P}[1]{>{\centering\arraybackslash}p{#1}}
\DeclareCaptionFormat{upper}{#1#2\uppercase{#3}\par}

% Diseño de tabla para ingeniería
\captionsetup[table]{
	format=upper,
	name=TABLA,
	justification=centering,
	labelsep=period,
	width=.75\linewidth,
	labelfont=small,
	font=small,
}

%Diseño de tabla para ciencias sociales y humanidades
%\captionsetup[table]{
%	justification=raggedright,
%	labelsep=period,
%	labelfont=small,
%	singlelinecheck=false,
%	font={small,bf}
%}


% DISEÑO DE FIGURAS. Puedes elegir entre el estilo para ingeniería o para ciencias sociales y humanidades. Por defecto, está activado el estilo de ingeniería. Si deseas utilizar el otro, comenta las líneas del diseño de ingeniería y descomenta las del diseño de ciencias sociales y humanidades
\usepackage{graphicx}
\graphicspath{{imagenes/}} %ruta a la carpeta de imágenes

% Diseño de figuras para ingeniería
\captionsetup[figure]{
	format=hang,
	name=Fig.,
	singlelinecheck=off,
	labelsep=period,
	labelfont=small,
	font=small		
}

% Diseño de figuras para ciencias sociales y humanidades
%\captionsetup[figure]{
%	format=hang,
%	name=Figura,
%	singlelinecheck=off,
%	labelsep=period,
%	labelfont=small,
%	font=small		
%}


% NOTAS A PIE DE PÁGINA
\usepackage{chngcntr} %para numeración contínua de las notas al pie
\counterwithout{footnote}{chapter}

% LISTADOS DE CÓDIGO
% soporte y estilo para listados de código. Más información en https://es.wikibooks.org/wiki/Manual_de_LaTeX/Listados_de_código/Listados_con_listings
\usepackage{listings}

% definimos un estilo de listings
\lstdefinestyle{estilo}{ frame=Ltb,
	framerule=0pt,
	aboveskip=0.5cm,
	framextopmargin=3pt,
	framexbottommargin=3pt,
	framexleftmargin=0.4cm,
	framesep=0pt,
	rulesep=.4pt,
	backgroundcolor=\color{gray97},
	rulesepcolor=\color{black},
	%
	basicstyle=\ttfamily\footnotesize,
	keywordstyle=\bfseries,
	stringstyle=\ttfamily,
	showstringspaces = false,
	commentstyle=\color{gray45},     
	%
	numbers=left,
	numbersep=15pt,
	numberstyle=\tiny,
	numberfirstline = false,
	breaklines=true,
	xleftmargin=\parindent
}

\captionsetup[lstlisting]{font=small, labelsep=period}
% fijamos el estilo a utilizar 
\lstset{style=estilo}
\renewcommand{\lstlistingname}{\uppercase{Código}}


%BIBLIOGRAFÍA - PUEDES ELEGIR ENTRE ESTILO IEEE O APA. POR DEFECTO ESTÁ CONFIGURADO IEEE. SI DESEAS USAR APA, COMENTA LAS LÍNEA DE IEEE Y DESCOMENTA LAS DE APA. Si haces cambios en la configuración de la bibliografía y no obtienes los resultados esperados, es recomendable limpiar los archivos auxiliares y volver a compilar en este orden: COMPILAR-BIBLIOGRAFIA-COMPILAR
% Tienes más información sobre cómo generar bibliografía en http://tex.stackexchange.com/questions/154751/biblatex-with-biber-configuring-my-editor-to-avoid-undefined-citations , https://es.sharelatex.com/learn/Bibliography_management_in_LaTeX y en http://www.ctan.org/tex-archive/macros/latex/exptl/biblatex-contrib
% También te recomendamos consultar la guía temática de la Biblioteca sobre citas bibliográficas: http://uc3m.libguides.com/guias_tematicas/citas_bibliograficas/inicio

% CONFIGURACIÓN PARA LA BIBLIOGRAFÍA IEEE
\usepackage[backend=bibtex, style=ieee, isbn=false,sortcites, maxbibnames=5, minbibnames=1]{biblatex} % Configuración para el estilo de citas de IEEE, recomendado para el área de ingeniería. "maxbibnames" indica que a partir de 5 autores trunque la lista el primero (minbibnames) y añada "et al." tal y como se utiliza en el estilo IEEE.

%CONFIGURACIÓN PARA LA BIBLIOGRAFÍA APA
%\usepackage[style=apa, backend=biber, natbib=true, hyperref=true, uniquelist=false, sortcites]{biblatex}
%\DeclareLanguageMapping{spanish}{spanish-apa}

% Añadimos las siguientes indicaciones para mejorar la adaptación de los estilos en español
\DefineBibliographyStrings{spanish}{%
	andothers = {et\addabbrvspace al\adddot}
}
\DefineBibliographyStrings{spanish}{
	url = {\adddot\space[En línea]\adddot\space Disponible en:}
}
\DefineBibliographyStrings{spanish}{
	urlseen = {Acceso:}
}
\DefineBibliographyStrings{spanish}{
	pages = {pp\adddot},
	page = {p.\adddot}
}

\addbibresource{bibliografia/bibliografia.bib} % llama al archivo bibliografia.bib que utilizamos de ejemplo


%-------------
%	DOCUMENTO
%-------------

\begin{document}
\pagenumbering{roman}
	
%----------
%	PORTADA
%----------	
\begin{titlepage}
	\begin{sffamily}
	\color{azulUC3M}
	\begin{center}
		\begin{figure}[H] %incluimos el logotipo de la Universidad
			\makebox[\textwidth][c]{\includegraphics[width=16cm]{Portada_Logo.png}}
		\end{figure}
		\vspace{2.5cm}
		\begin{Large}
			Grado Ingeniería de Sistemas Audiovisuales\\			
			2018-2019\\
			\vspace{2cm}		
			\textsl{Trabajo Fin de Grado}
			\bigskip
			
		\end{Large}
		 	{\Huge ``Diseño e implementación de un microservicio con Spring''}\\
		 	\vspace*{0.5cm}
	 		\rule{10.5cm}{0.1mm}\\
			\vspace*{0.9cm}
			{\LARGE Jesús Rienda Iáñez}\\ 
			\vspace*{1cm}
		\begin{Large}
			Tutor/es\\
			Carmen Pelaez Moreno\\
			Leganés, 2019\\
		\end{Large}
	\end{center}
	\vfill
	\color{black}
	\includegraphics[width=4.2cm]{imagenes/creativecommons.png}\\
	\emph{[Incluir en el caso del interés en su publicación en el archivo abierto]}\\
	Esta obra se encuentra sujeta a la licencia Creative Commons \textbf{Reconocimiento - No Comercial - Sin Obra Derivada}
	\end{sffamily}
\end{titlepage}

\newpage %página en blanco o de cortesía
\thispagestyle{empty}
\mbox{}

%----------
%	RESUMEN Y PALABRAS CLAVE
%----------	
\renewcommand\abstractname{\large\bfseries\filcenter\uppercase{Resumen}}
\begin{abstract}
\thispagestyle{plain}
\setcounter{page}{3}
	
	En este trabajo fin de grado se ha realizado un estudio sobre las posibilidades existentes en el desarrollo de aplicaciones web. Desde el desarrollo partiendo de cero hasta el uso de software de terceros como son los framework para facilitar el desarrollo y centrarse en la funcionalidad.
	
	Para el correcto diseño de nuestra arquitectura ha sido necesario descomponer e investigar cada integrante de esta. Por una parte se ha investigado sobre los patrones de diseño existentes y sus funcionalidades, también sobre tipos de arquitecturas, framework y ademas sobre las bases de datos.
	
	Para afianzar todo lo desarrollado en este trabajo se ha implementado un microservicio basado en listas de reproducción de Spotify. Utilizando para ello las ultimas tecnologías de desarrollo como Spring Boot, Spring Data y una base de datos muy potente como MongoDB. Este microservicio hace de intermediario con la base de datos, filtrando por ejemplo las listas de reproducción que contienen canciones que ha escuchado un usuario.
	
	\textbf{Palabras clave:}
	Microservicio, aplicaciones web, SOA, API, Swagger, MongoDb, Lombok, Spring Boot
	
	\vfill
\end{abstract}
	\newpage %página en blanco o de cortesía
	\thispagestyle{empty}
	\mbox{}


%----------
%	DEDICATORIA
%----------	
\chapter*{Dedicatoria}

\setcounter{page}{5}
	
	% ESCRIBIR LA DEDICATORIA AQUÍ	
		
	\vfill
	
	\newpage %página en blanco o de cortesía
	\thispagestyle{empty}
	\mbox{}
	

%----------
%	ÍNDICES
%----------	

%--
%Índice general
%-
\tableofcontents
\thispagestyle{fancy}

\newpage %página en blanco o de cortesía
\thispagestyle{empty}
\mbox{}

%--
%Índice de figuras. Si no se incluyen, comenta las líneas siguientes
%-
\listoffigures
\thispagestyle{fancy}

\newpage %página en blanco o de cortesía
\thispagestyle{empty}
\mbox{}

%--
%Índice de tablas. Si no se incluyen, comenta las líneas siguientes
%-
\listoftables
\thispagestyle{fancy}

\newpage %página en blanco o de cortesía
\thispagestyle{empty}
\mbox{}


%----------
%	TRABAJO
%----------	
\clearpage
\pagenumbering{arabic} % numeración con múmeros arábigos para el resto de la publicación	

\chapter{Resumen}
\section{Introducción}
	% COMENZAR A ESCRIBIR EL TRABAJO
	El mundo de la tecnología se basa en tendencias, hoy en día los microservicios están de moda, gracias a grandes compañías como Amazon, Ebay, Twitter, Paypal, Netflix, etc. Pero realmente es una tecnología que ha surgido gracias a la aparición de la nube(cloud) donde todas estas empresas se están llevando sus software
	
	En la nube no necesitas servidores físicos, con el mantenimiento y la inversión que esto supone, sino que solo pagas por los recursos que necesitas en cada momento. Esto es una gran revolución para la informática, te permite gestionar los picos de actividad sin perder servicio.
	
	
	\subsection{Planteamiento del problema}
	Contamos con datos reales de Spotify que contienen información de las listas de reproducción.
	Tenemos también datos de usuarios con sus canciones.

	Necesitamos una base de datos donde almacenar todos los parámetros y vamos a crear un programa para realizar las consultas.	
		
	Este programa contará con tiempos de respuesta bajos ya que podrán consultarlo muchas personas a la vez.
		
	\subsection{Objetivos}
		
	Diseñar una arquitectura de software adecuada para el problema.
	
	Crear una base datos con los ficheros de listas.
	
	Implementar un microservicio.
	
	
	\subsection{Solución propuesta}
	Como solución a nuestro problema vamos a diseñar una arquitectura de microservicios.
	Implementamos un microservicio con java mediante el framework Spring Boot y para llamar a este utilizaremos Swagger.
	La BBDD estará almacenada en Azure y sera CosmosDB
	\section{Estado del arte}
	
	\section{Justificación de la solución}
	
	
	Elegimos arquitectura de microservicios ya que ofrecen escalabilidad y alta disponibilidad.
	
	Para el desarrollo elegimos Spring Boot que cuenta con las ventajas de Sprint y es el framework que mejor se adapta a nuestro proyecto ya que es el más antiguo y de mayor uso.
	
	Para el desarrollo elegimos Spring Boot, no solo nos aporta ventajas de cualquier framwork de microservicios sino que también tenemos todas la funcionalidades de Spring como la conexión con base de datos y la arquitectura en capas.
	
	Para la gestión de dependencias elegimos Maven ya que esta dirigido para java y nos va a empaquetar la aplicación.
	
	Utilizamos Lombok para evitar código repetitivo. Ademas ofrece una implementación del patrón de diseño builder.
	
	La comunicación con el microservicio va a ser mediante REST, protocolo simple y eficaz para realizar las distintas operaciones.
	
	Todo servicio REST necesita un documento(API) donde se describa la entrada salida de él, en este caso vamos a utilizar anotaciones de 
	
	Utilizamos Swagger para autogenerar el documento API en el que se definen las entradas y salidas de todos los servicios desarrollados.
	
	Para almacenar los datos con los que contamos la mejor opción es MongoDB, con esta no es necesario adaptar los datos ya que se guardan con el formato actual. Para que sea más eficiente pondremos nuestra BBDD en la nube y utilizaremos CosmosDB.
	
	
	\section{Desarrollo de la aplicación}
	
	\section{Marco regulador}
	
	No existen leyes que se apliquen en el desarrollo de un microservicio. Sin embargo si se trata con datos reales hay que tener en cuenta dos leyes:
	\begin{itemize}
		\item Ley Orgánica de Protección de Datos Personales y garantía de los derechos digitales.
		\item Ley General de Telecomunicaciones.
	\end{itemize}
	
	
	\section{Entorno socio-económico}	
	
	\subsection{Presupuesto}
	
	
	El proyecto se ha desarrollado durante 8 meses con un total de 300 horas empleadas lo que nos supone un total de 4500 euros. 
	Los servicios contratados son Azure CosmosDB que nos supone un total de 1069,34 euros.
	En los costes totales del proyecto sumamos la mano de obra y los servicios contratados y nos saldría un total de 5569,34 euros.
	
	
	\subsection{Impacto socio-económico}
	
	Uno de los objetivos de desarrollar aplicaciones con microservicios es que estas sean fácilmente mantenibles y que puedan evolucionar según las necesidades de la sociedad.
	
	Una de las ventajas de implantar los microservicios en la nube es la mejor utilización de los recursos disponibles para aprovecharlos de manera eficiente abasteciéndose con tecnologías renovables.
	
	Cuando se esperaba un numero grande de peticiones era necesario un gran número de servidores que suponía recursos sin utilizar una vez dado el servicio, esto lo solucionan los microservicios en la nube.
	
	Un inconveniente de esto es para una aplicación pequeña el coste es más elevado.
	
	\section{Conclusiones}
	
	Hemos cumplido el objetivo de diseñar una arquitectura capaz de recibir multitud de peticiones.
	
	Para la base de datos elegimos una en la nube, aunque esta es la mejor opción, nos hemos dado cuenta que es muy caro para una pequeña aplicación ya que cobran por uso y almacenamiento. Sin embargo si que hemos podido ver que para el problema que teníamos, la base de datos elegida es ideal.
	
	Implementar el microservicio ha sido una tarea de investigación ya que todas las tecnologías utilizadas son bastante novedosas y es complicado encontrar información al respecto. Pero gracias a ello hemos conseguido una aplicación limpia y fácilmente mantenible.
	
	\subsection{Trabajos futuros}
	
	Posibles trabajos futuros que podrían mejorar o complementar nuestra aplicación.
	
	-Desarrollar una aplicación web para visualizar los datos.
	
	-Añadir nuevas funcionalidades sobre los datos existentes.
	
	-Dockerizar (contenerizar) la aplicación para integrarla en un contenedor y desplegarlo en la nube.
	
	-Incluir un gestor de contenedores en Azure para gestionar el escalado y otras configuraciones de las aplicaciones que tengamos dockerizadas.
	
	\section{Estado del arte}

	En la década de los 60 surgió la arquitectura de software, pero hasta 1980 no se integro totalmente en el desarrollo de software. 
	
	El Instituto de Ingeniera de Software la define como:
	"La arquitectura de software es una representación del sistema que ayuda a comprender cómo se comportará un programa." 
	
	La arquitectura es la principal portadora de cualidades del sistema, como el rendimiento, la modificabilidad y la seguridad y no se puede lograr sin una visión arquitectónica unificadora.
	
	El sistema se divide en elementos de software llamados módulos. Las propiedades de estos elementos pueden ser de dos tipos, internas y externas.
	
	Las \textbf{propiedades internas}  son las que definen el lenguaje en el que está desarrollado y todos los detalles de la implementación, como pueden ser:
	
	- Entidades dinámicas en tiempo de ejecución como objetos e hilos.
	
	- Entidades lógicas en tiempo de desarrollo como clases y módulos.
	
	- Entidades físicas como nodos o carpetas.	
	
	Las \textbf{propiedades externas} son los contratos que existen entre módulos y que permiten a otros módulos establecer dependencias/conexiones entre ellos.
	
	Existen patrones de diseño, arquitectónicos y frameworks que nos proponen soluciones a nuestro problema.
	
	\subsection{Patrones de diseño}
	
	Los patrones en la arquitectura de software no aparecieron hasta 1994 en un libro llamado "Patrones de diseño" que a día de hoy sigue siendo un modelo de referencia.		 
	Los patrones definidos en el libro se dividen en función del problema a resolver:
	 
	 \begin{itemize}
	 	\item Patrones creacionales: utilizados para facilitar la creación de nuevos objetos. Los más usados son Singleton y Builder
	 	\item Patrones estructurales: facilitan la modelización de nuestro software, definiendo la forma en que las clases se relacionan entre si. Los más conocidos son Facade, Decorator y Proxy.
 	 	\item Patrones de comportamiento: gestionan algoritmos, relaciones y responsabilidades entre objetos. El más conocido es Strategy 	
	 \end{itemize}
 
 \subsection{Patrones de arquitectura}
 Estos tienen un alto nivel de abstraccion. Estos al igual que los patrones de diseño, solucionan problemas recurrentes de una forma reutilizable.
 
 Los patrones de arquitectura mas usados son:
 	\begin{itemize}
 	\item Programación por capas: Estructurar programas que pueden descomponerse en subtareas. 
 	\item Cliente-servidor: Cuenta con un servidor y múltiples clientes, el servidor será el que de servicio a diversos componentes del cliente y los clientes solicitarán servicio al servidor.
 	\item Arquitectura orientada a servicios: Da soporte mediante la creación de servicios. 
 	\item Arquitectura de microservicios: Crear aplicaciones usando un conjunto de pequeños servicios que se comunican entre sí pero se ejecutan de forma individual.
 	\item Pipeline: Oorganizar sistemas que procesan una sucesión de datos que pasan a través de tuberías.
 	\item Arquitectura en pizarra: Para la resolución de problemas que se desconoce su estrategia. Está formado por tres componentes:
 	\begin{itemize}
 		\item \textbf{Pizarra:} memoria que contiene todos los objetos. 
 		\item \textbf{Fuente de conocimiento:} módulos especializados. 
 		\item \textbf{Componente de control:} selecciona y ejecuta los módulos.
 	\end{itemize}
 	\item Arquitectura dirigida por eventos: Maneja principalmente los eventos y está formado por cuatro componentes que son: fuente de evento , escucha de evento , canal y bus de evento.
 	\item Peer-to-peer: Se llama pares a las componentes individuales y estos pueden funcionar como servidor, dando servicio a otros pares, o como cliente, pidiendo servicio a otros pares.
 	\item Modelo Vista Controlador: Divide un programa interactivo en tres partes:
 		\begin{itemize}
 			\item \textbf{Modelo:} formado por los datos básicos y la funcionalidad del programa. 
 			\item \textbf{Vista:} maneja la visualización de la información. 
 			\item \textbf{Controlador:} controla la entrada (teclado y ratón) e informa al modelo y la vista de los cambios.
 		\end{itemize}
 
 \end{itemize}

\section{Frameworks}
  Estos son estructuras de software ya implementadas en las que un programador puede apoyarse para desarrollar un proyecto propio. Los frameworks están implementados y siguen tanto los patrones de sotfware como los patrones de arquitectura y suelen incluir ficheros de configuración, librerías, etc.
 
 Existen frameworks para prácticamente todos los lenguajes de programación pero es importante valorar y elegir el framework que mejor se adapta al problema que vas a resolver, por ejemplo Angular orientado al desarrollo del front-end o Spring orientado al desarrollo de back-end.
 
 \subsection{Frameworks de java para microservicios}
 Encontramos los siguientes orientados a microservicios.
 \begin{itemize}
 	\item Spring Boot: ofrece un arquetipo de un microservicio básico, ampliable fácilmente con poco desarrollo. 
 	\item Cricket: framework para implementación de arquitectura dirigida por eventos.
 	\item Dropwizard: framework para arquitectura de microservicios por capas.
 	\item Eclipse MicroProfile: arquetipo de la aplicación con las dependencias necesarias.
 	\item Helidon: Oracle lo utilizó internamente bajo el nombre "Java for Cloud" y finalmente se ha liberado para uso público. 
 \end{itemize}
	
	\section{Bases de datos}
	Para el almacenamiento de los datos que vamos a utilizar bases de datos que se pueden definir como un conjunto de información relacionada que se encuentra agrupada ó estructurada. Son sistemas formados por grupos de datos que permiten la consulta, modificación y borrado de los mismos. Se crearon para poder almacenar grandes cantidades de información. 
	
	Actualmente dominan el mercado de las bases de datos IBM, Microsoft y Oracle y en el campo de internet Google.
	
	Más tarde surgieron las bases de datos no relacionales(NoSQL) que son las que no contienen un identificador para relacionar un conjunto de datos con otro.	
	La más exitosa en bases de datos no relacionales es MongoDB seguida por Redis, Elasticsearch y Cassandra.
	
	\subsection{Pros y contras bases de datos relacionales}
	Ventajas:
	 \begin{itemize}
		\item Gran variedad de información para poder realizar cualquier consulta.
		\item La información no se completa si a mitad de una operación ocurre un problema.
		\item Tiene los estándares bien definidos.
		\item Es sencillo a la hora de programar.
	\end{itemize}
	Inconvenientes:
		 \begin{itemize}
		\item Crecen demasiado y aumenta el tiempo de respuesta.
		\item	Se tendrá que reestructurar si se realiza cualquier cambio.
		\item 	No garantizan el buen funcionamiento si el sistema operativo no cumple los requerimientos mínimos.
	\end{itemize}

	\subsection{Pros y contras bases de datos no relacionales}
		Ventajas:
	\begin{itemize}
		\item Adaptabilidad para crecimientos o cambios.
		\item Crecimiento horizontal. 
		\item No es necesario contar con servidores de gran cantidad.
		\item Cuenta con un algoritmo interno que reformule las consultas para no sobrecargar los servidores.
	\end{itemize}
	Inconvenientes:
	\begin{itemize}
		\item No garantiza que si la operación falla se complete la información.
		\item	Se necesitan conocimientos elevados en el uso de esta herramienta pues las operaciones son limitadas.
		\item 	No tiene un estándar de lenguaje definido.
		\item 	No contiene una interfaz gráfica por lo que es necesario hacer todo mediante consola.
	\end{itemize}

\section{Microservicios}

Son útiles para desarrollar una aplicación, dividiéndola en una serie de pequeños servicios, que se ejecutan de forma independiente y se comunican entre sí.

Tiene que haber un número mínimo de servicios encargados de gestionar los procesos en común. Cada microservicio es pequeño e independiente y se corresponde con un área de la aplicación. 

\subsection{Pros y contras de los microservicios}
Ventajas:
\begin{itemize}
	\item Libertad de implementar y desplegar los servicios de forma independiente.
	\item Los microservicios se pueden desarrollar con un equipo de trabajo mínimo.
	\item Posibilidad de utilizar lenguajes diferentes.
	\item Con el uso de contenedores el desarrollo y despliegue son más rápidos.
	\item Fácilmente escalables.
\end{itemize}
Inconvenientes:
\begin{itemize}
	\item Pruebas complicados por el despliegue distribuido.
	\item Puede dar lugar a grandes bloques de información para gestionar.
	\item Con un gran número de servicios, integrarlos y gestionarlos puede resultar complicado.
	\item Alto consumo de memoria.
	\item Fragmentar una aplicación en microservicios puede llevar muchas horas de planificación.
\end{itemize}

\subsection{Ejemplos de empresas que utilizan microservicios}


Encontramos estas compañías Netflix, Amazon y Ebay.


\chapter{Desarrollo de la aplicación}

El primer es tener instalado Eclipse, de aquí vamos a la web https://start.spring.io/ usando el framework Spring Boot. Elegimos Maven para el control de dependencias, escogemos java como lenguaje y por ultimo las dependencias que queremos incluir, MongoDB, Spring Web, Lombok.

Swagger es un software para diseñar, documentar y consumir servicios Rest.Para configurar Swagger es necesario añadir una clase de configuración con la anotacción @EnableSwagger2 para habilitar la generación del dominio.

Creamos un repositorio en github (https://github.com/jesusRienda/microservicio) para poder guardar lo que tenemos hasta ahora.

El arquetipo contiene una clase App que hay que ejecutar para levantar el microservicio en local, para hacerlo al tener una dependencia de MongoDb tenemos que establecer la conexión con la BBDD.

 Crear en azure una instancia de CosmosDb y configurar el microservicio para que apunte a esa base de datos. Creamos una clase java llamada MongoDbConfig y mediante la anotación @Value obtenemos de application.properties la conexión a cosmos y el nombre de la BBDD. Una vez hecho esto ya podemos levantar el microservicio.

 Lo siguiente es crear las tres capas que vamos a utilizar, la capa de persistence, donde se definen las querys, la capa de service donde se implementa la lógica de negocio y la capa controller donde se implementan los verbos http mediante los que se va a llamar al servicio.
 
 Una vez tenemos los métodos necesarios en la capa de persistencia pasamos a la capa de servicio en la que vamos a crear una interfaz con los métodos que vamos a exponer a la capa de presentación y una implementación de esa interfaz. 
 
 En esta capa se llaman a los métodos necesarios de la capa anterior y además puedes llamar a métodos de un servicio diferente.
 
 La capa de presentación es la encargada de serializar un json a un objeto y en función del recurso, llamar a un service u otro. En ella se definen los recursos y los verbos de la petición. 
 




%----------
%	BIBLIOGRAFÍA
%----------	

%\nocite{*} % Si quieres que aparezcan en la bibliografía todos los documentos que la componen (también los que no estén citados en el texto) descomenta está lína

\clearpage
\addcontentsline{toc}{chapter}{Bibliografía}
\setquotestyle[english]{british} % Cambiamos el tipo de cita porque en el estilo IEEE se usan las comillas inglesas.
\printbibliography



%----------
%	ANEXOS
%----------	

% Si tu trabajo incluye anexos, puedes descomentar las siguientes líneas
%\chapter* {Anexo x}
%\pagenumbering{gobble} % Las páginas de los anexos no se numeran




\end{document}